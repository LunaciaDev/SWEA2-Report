% Dokumententyp und benutzte Pakete
\documentclass[open=right,  % Kapitel darf nur auf rechten Seite beginnen
    paper=a4,               % DIN-A4-Papier
    fontsize=11pt,          % Schriftgöße
    headings=small,         % Kleine Überschriften
    headsepline=true,       % Trennlinie am Kopf der Seite
    footsepline=false,      % Keine Trennlinie am Fuß der Seite
    bibliography=totoc,     % Literaturverzeichnis in das Inhaltsverzeichnis aufnehmen
    twoside=true,             % Doppelseitiger Druck - auf off stellen für einseitig
    DIV=7,                  % Verhältnis der Ränder zum bedruckten Bereich
    chapterprefix=true,     % Kapitel x vor dem Kapitelnamen
    cleardoublepage=plain]{scrbook}

% Pakete einbinden, die benötigt werden
\usepackage{ifthen}               % Logische Bedingungen mit ifthenelse
\usepackage{scrlayer-scrpage}     % Erweiterte Einstellungen an scrbook zulassen
\usepackage[utf8]{inputenc}       % Dateien in UTF-8 benutzen
\usepackage[T1]{fontenc}          % Zeichenkodierung
\usepackage{graphicx}             % Bilder einbinden
\usepackage{enumitem}             % Eigene Listen definieren können
\usepackage{setspace}             % Abstände korrigieren

\usepackage[main=english, ngerman]{babel} % Englische Sprachunterstützung
\usepackage[autostyle=true,english=american]{csquotes} % Englische Anführungszeichen
\usepackage[pagebackref=false,english]{hyperref}  % Hyperlinks
\newcommand{\frauassortlocale}{en_US} % Sortierung der Literatur

\usepackage{xcolor}               % Unterstützung für Farben
\usepackage{amsmath}              % Mathematische Formeln
\usepackage{amsfonts}             % Mathematische Zeichensätze
\usepackage{amssymb}              % Mathematische Symbole
\usepackage{float}                % Fließende Objekte (Tabellen, Grafiken etc.)
\usepackage{booktabs}             % Korrekter Tabellensatz
\usepackage[printonlyused]{acronym}  % Abkürzungsverzeichnis [nur verwendete Abkürzungen]
\usepackage{makeidx}              % Sachregister
\usepackage{listings}             % Quelltexte
\usepackage{listingsutf8}         % Quelltexte in UTF8
\usepackage[hang,font={sf,footnotesize},labelfont={footnotesize,bf}]{caption} % Beschriftungen
\usepackage[scaled]{helvet}       % Schrift Helvetia laden
\usepackage[absolute]{textpos}    % Absolute Textpositionen (für Deckblatt)
\usepackage{calc}                 % Berechnung von Positionen
\usepackage{blindtext}            % Blindtexte
%\usepackage[bottom=40mm,left=30mm,right=30mm,top=35mm]{geometry} % Ränder ändern
\usepackage[ 											% Ränder ändern
	a4paper, 
	showframe=false,
	textwidth=420pt,
	textheight=620pt,
	footskip=40pt,
]{geometry}
\usepackage{scrhack}              % tocbasic Warnung entfernen
\usepackage[all]{hypcap}          % Korrekte Verlinkung von Floats
\usepackage{tabularx}             % Spezielle Tabellen
\usepackage[
  backend       = bibtex8, %minalphanames only works with biber backend
  bibstyle      = numeric-comp,
  citestyle     = numeric-comp,
  firstinits    = true,
  useprefix     = true, %print "von, van, etc.", too.
  minnames      = 1,
  minalphanames = 3,
  maxalphanames = 4,
  maxbibnames   = 99,
  maxcitenames  = 3,
	url						= false,
  doi           = true, %source: http://tex.stackexchange.com/a/23118/9075
  isbn          = true, %source: http://tex.stackexchange.com/a/23118/9075
	eprint				= false,
  backref       = true]{biblatex}
\usepackage{rotating}             % Seiten drehen
\usepackage{harveyballs}          % Harveyballs
\usepackage{chngcntr}             % Counter (Zähler) ändern können - für Fußnotennummern

% Einstellungen zu den Fußnoten
\renewcommand{\footnotesize}{\fontsize{9}{10}\selectfont} % Größe der Fußnoten
\setlength{\footnotesep}{8pt} % Abstand zwischen den Fußnoten

% Kommentieren Sie diese Zeile ein, wenn Sie eine "durchlaufende" Nummerierung bei den
% Fußnoten wünschen, d.h. wenn die Fußnoten nicht bei jedem Kapitel wieder bei 1
% beginnen sollen.
%\counterwithout{footnote}{chapter}

\setlength{\bibitemsep}{1em}      % Abstand zwischen den Literaturangaben
\setlength{\bibhang}{2em}         % Einzug nach jeweils erster Zeile

% Trennung von URLs im Literaturverzeichnis (große Werte [> 10000] verhindern die Trennung)
\defcounter{biburlnumpenalty}{10} % Strafe für Trennung in URL nach Zahl
\defcounter{biburlucpenalty}{500} % Strafe für Trennung in URL nach Großbuchstaben
\defcounter{biburllcpenalty}{500} % Strafe für Trennung in URL nach Kleinbuchstaben

% Farben definieren
\definecolor{linkblue}{RGB}{0, 0, 100}
\definecolor{linkblack}{RGB}{0, 0, 0}
\definecolor{comment}{RGB}{63, 127, 95}
\definecolor{darkgreen}{RGB}{14, 144, 102}
\definecolor{darkblue}{RGB}{0,0,168}
\definecolor{darkred}{RGB}{128,0,0}
\definecolor{javadoccomment}{RGB}{0,0,240}

% Einstellungen für das Hyperlink-Paket
\hypersetup{
    colorlinks=true,      % Farbige links verwenden
%    allcolors=linkblue,
    linktoc=all,          % Links im Inhaltsverzeichnis
    linkcolor=linkblack,  % Querverweise
    citecolor=linkblack,  % Literaturangaben
    filecolor=linkblack,  % Dateilinks
    urlcolor=linkblack    % URLs
}

% Einstellungen für Quelltexte
\lstset{
      xleftmargin=0.2cm,
      basicstyle=\footnotesize\ttfamily,
      keywordstyle=\color{darkgreen},
      identifierstyle=\color{darkblue},
      commentstyle=\color{comment},
      stringstyle=\color{darkred},
      tabsize=2,
      lineskip={2pt},
      columns=flexible,
      inputencoding=utf8,
      captionpos=b,
      breakautoindent=true,
      breakindent=2em,
      breaklines=true,
      prebreak=,
      postbreak=,
      numbers=none,
      numberstyle=\tiny,
      showspaces=false,      % Keine Leerzeichensymbole
      showtabs=false,        % Keine Tabsymbole
      showstringspaces=false,% Leerzeichen in Strings
      morecomment=[s][\color{javadoccomment}]{/**}{*/},
      literate={Ö}{{\"O}}1 {Ä}{{\"A}}1 {Ü}{{\"U}}1 {ß}{{\ss}}2 {ü}{{\"u}}1 {ä}{{\"a}}1 {ö}{{\"o}}1
}

\urlstyle{same}

% Einstellungen für Überschriften
\renewcommand*{\chapterformat}{%
  \LARGE\chapapp~\thechapter
  \vspace{0.3cm}
} 
% Abstände für die Überschriften setzen
\renewcommand{\chapterheadstartvskip}{\vspace*{2.6cm}}
\renewcommand{\chapterheadendvskip}{\vspace*{1.5cm}}

% Vertikale Abstände für die Überschriften etwas verkleinern
\RedeclareSectionCommand[
  beforeskip=-1.0\baselineskip,
  afterskip=0.25\baselineskip]{section}

\RedeclareSectionCommand[
  beforeskip=-1.0\baselineskip,
  afterskip=0.15\baselineskip]{subsection}

\RedeclareSectionCommand[
  beforeskip=-1.0\baselineskip,
  afterskip=0.15\baselineskip]{subsubsection}

% In der Kopfzeile nur die kurze Kapitelbezeichnung (ohne Kapitel davor)
\renewcommand*\chaptermarkformat{\thechapter\autodot\enskip}
\automark[chapter]{chapter}

% Einstellungen für Schriftarten
\setkomafont{pagehead}{\normalfont\sffamily}
\setkomafont{pagenumber}{\normalfont\sffamily}
\setkomafont{paragraph}{\sffamily\bfseries\small}
\setkomafont{subsubsection}{\sffamily\itshape\bfseries\small}
\addtokomafont{footnote}{\sffamily\footnotesize\sffamily}
\setkomafont{chapter}{\Huge\selectfont\bfseries}
\renewcommand*{\bibfont}{\normalsize\sffamily}

% Wichtige Abstände
\setlength{\parskip}{0.0cm}  % 2mm Abstand zwischen zwei Absätzen
\setlength{\parindent}{1.5em}  % Absätze nicht einziehen
\clubpenalty = 10000         % Keine "Schusterjungen"
\widowpenalty = 10000        % Keine "Hurenkinder"
\displaywidowpenalty = 10000 % Keine "Hurenkinder"
                             % Siehe: https://de.wikipedia.org/wiki/Hurenkind_und_Schusterjunge
\pdfminorversion=5 
\pdfcompresslevel=9
\pdfobjcompresslevel=2

% Index erzeugen
\makeindex

% Einfacher Font-Wechsel über dieses Makro
\newcommand{\changefont}[3]{
\fontfamily{#1} \fontseries{#2} \fontshape{#3} \selectfont}

% Eigenes Makro für Bilder. Das label (für \ref) ist dann einfach
% der Name der Bilddatei
\newcommand{\image}[2]{
\begin{figure}[ht]
  \centering
  \includegraphics{#1}
  \caption{#2}
  \label{#1}
\end{figure}}

% Wo sind die Bilder?
\graphicspath{{bilder/}}

% Makros für typographisch korrekte Abkürzungen
\newcommand{\eg}[0]{e.\,g.}
\newcommand{\ie}[0]{i.\,e.}
\newcommand{\ea}[0]{et\,al.}

% Flags für Veröffentlichung und Sperrvermerk
\newboolean{frauaspublizieren}
\newboolean{frauassperrvermerk}

% Tabellenzellen mit mehreren Zeilen
\newcolumntype{L}{>{\raggedright\arraybackslash}X}
\newcolumntype{b}{l}
\newcolumntype{s}{>{\hsize=.3\hsize}l}
\newcolumntype{F}{>{\hsize=\dimexpr2\hsize+2\tabcolsep+\arrayrulewidth\relax}X}

% Checklisten mit zwei Ebenen
\newlist{checklist}{itemize}{2}
\setlist[checklist]{label=$\square$}

% Dokumenteninfos importieren
% -------------------------------------------------------
% Daten für die Arbeit
% Wenn hier alles korrekt eingetragen wurde, wird das Titelblatt
% automatisch generiert. D.h. die Datei titelblatt.tex muss nicht mehr
% angepasst werden.

% Work Title in English
\newcommand{\frauastitelen}{Application of a Flux Compensator for Timetravel with a Maximum Velocity of Warp~7}

% Weitere Informationen zur Arbeit
\newcommand{\frauasort}{Frankfurt am Main}    % Ort
\newcommand{\frauasautor}{Duc Xuan Bui, Pham Minh Tuan Bui, Hai Duong Tran, Hao Nhien Truong, Duc An Pham, Hoang Anh Khoa Nguyen}
\newcommand{\frauasdatum}{05.02.2025} % Datum der Abgabe
\newcommand{\frauasjahr}{2025} % Jahr der Abgabe
\newcommand{\frauasfirma}{} % Firma bei der die Arbeit durchgeführt wurde
\newcommand{\frauasbetreuer}{Prof. Peter Mustermann, University of Applied Sciences} % Betreuer an der Hochschule
\newcommand{\frauaszweitkorrektor}{Erika Mustermann, Paukenschlag GmbH} % Betreuer im Unternehmen oder Zweitkorrektor
\newcommand{\frauasfakultaet}{I} % I für Informatik oder E, S, B, D, M, N, W, V
\newcommand{\frauasstudiengang}{IB} % IB IMB UIB CSB IM MTB (weitere siehe titleblatt.tex)

% Zustimmung zur Veröffentlichung
\setboolean{frauaspublizieren}{true}   % Einer Veröffentlichung wird zugestimmt
\setboolean{frauassperrvermerk}{false} % Die Arbeit hat keinen Sperrvermerk

% Literatur-Datenbank
\bibliography{bibliography}
%\addbibresource{bibliography.bib}   % BibLaTeX-Datei mit Literaturquellen einbinden
\hyphenation{
multi-screen
}

% Anfang des Dokuments
\begin{document}
% Die Seitennummerierung erfolgt durchlaufend ab der Titelseite. Also keine
% Spielereien mit römischen Ziffern usw. - Die ISO 7144 schreibt das sogar für
% wissenschaftliche Werke vor.
% Von Promitionsordnung verlangt!
% Deshalb ist \frontmatter DEAKTIVIERT

% Römische Ziffern für die "Front-Matter"
\setcounter{page}{0}
%\changefont{ptm}{m}{n}  % Times New Roman für den Fließtext
%\renewcommand{\rmdefault}{ptm}

% Titelblatt
% *******************************************************************
% In dieser Datei sollten eigentlich keine Veränderungen
% notwendig sein. Alle Einstellungen erfolgen in docinfo.tex und
% der thesis.tex.
% *******************************************************************

\thispagestyle{empty}

% Fakultäten 
% *******************************************************************
\ifthenelse{\equal{\frauasfakultaet}{I}}%
  {\newcommand{\frauasfakultaetlangde}{Fachbereichs Informatik und Ingenieurwissenschaften}%
   \newcommand{\frauasfakultaetlangen}{Faculty of Computer Science and Engineering}}{} 

% Studiengänge 
% *******************************************************************
\ifthenelse{\equal{\frauasstudiengang}{IB}}%
  {\newcommand{\frauasstudienganglangde}{Informatik}%
  \newcommand{\frauasstudienganglangen}{Computer Science}%
  \newcommand{\frauastypde}{Bachelor-Thesis}%
  \newcommand{\frauastypen}{Bachelor Thesis}%
  \newcommand{\frauasgrad}{\frauasbsc}}{} 
	
% Abschlüsse
% *******************************************************************
\newcommand{\frauasbsc}{Bachelor of Science (B.Sc.)}
\newcommand{\frauasba}{Bachelor of Arts (B.A.)}
\newcommand{\frauasmaster}{Master of Science (M.Sc.)}
\newcommand{\frauasmastera}{Master of Arts (M.A.)}
\newcommand{\frauasmasterba}{Master of Business Administration (MBA)}

\newcommand{\frauaskoerperschaftde}{Frankfurt University of Applied Sciences}
\newcommand{\frauaskoerperschaften}{Frankfurt University of Applied Sciences}

\newcommand{\frauastyp}{\frauastypen}%
\newcommand{\frauasthesistype}{for the acquisition of the academic degree \frauasgrad}%
\newcommand{\frauaskoerperschaft}{\frauaskoerperschaften}%
\newcommand{\frauasstudiengangname}{Course of Studies: \frauasstudienganglang}%
\newcommand{\frauasstudienganglang}{\frauasstudienganglangen}%
\newcommand{\frauastitel}{\frauastitelen}%
\newcommand{\frauastutor}{Supervisor}
\newcommand{\frauasfakultaetlang}{\frauasfakultaetlangen}%
\newcommand{\frauaslistoftables}{List of Tables}%
\newcommand{\frauaslistoffigures}{List of Figures}%
\newcommand{\frauaslistings}{Listings}%
\newcommand{\frauasindex}{Index}%
\newcommand{\frauasabbreviations}{List of Abbreviations}%
\newcommand{\frauassnowcardanforderung}{Requirement}%
\newcommand{\frauassnowcardno}{\#}%
\newcommand{\frauassnowcardart}{Type}%
\newcommand{\frauassnowcardprio}{Prio}%
\newcommand{\frauassnowcardtitel}{Title}%
\newcommand{\frauassnowcardherkunft}{Origin}%
\newcommand{\frauassnowcardkonflikt}{Conflicts}%
\newcommand{\frauassnowcardbeschreibung}{Description}%
\newcommand{\frauassnowcardfitkriterium}{Fit Criterion}%
\newcommand{\frauassnowcardmaterial}{Supporting Material}%
\newcommand{\frauasqasanforderung}{QAS}%
\newcommand{\frauasqasno}{\#}%
\newcommand{\frauasqasart}{Type}%
\newcommand{\frauasqasprio}{Prio}%
\newcommand{\frauasqastitel}{Title}%
\newcommand{\frauasqasquelle}{Source}%
\newcommand{\frauasqasstimulus}{Stimulus}%
\newcommand{\frauasqasartefakt}{Artifact}%
\newcommand{\frauasqasumgebung}{Environment}%
\newcommand{\frauasqasantwort}{Response}%
\newcommand{\frauasqasmass}{Response Measure}%
\selectlanguage{english}

% Daten in die Standard-Felder von KOMA-Script eintragen
\titlehead{\frauastyp\ in\  \frauasstudienganglang}
\subject{}
\title{\frauastitel}
\author{\frauasauthor}
\date{\small{\frauasdatum}}

% Daten für das fertige PDF-Dokument
\hypersetup{
  pdftitle={\frauastitel},                           % Titel des Dokuments
  pdfauthor={\frauasautor},                          % Autor
  pdfsubject={\frauastyp\ in\ \frauasstudienganglang}, % Thema
  pdfkeywords={\frauastitel}                         % Schlüsselworte
}

\newlength{\bindekorrektur}
\newlength{\seitenanfang}
\newlength{\seitenbreite}

\setlength{\bindekorrektur}{-46mm}   % Korrektur der horizontalen Position
\setlength{\seitenanfang}{0mm}       % Korrektur der vertikalen Position 
 
% Thesis Titelseite 
\begin{center}
	\includegraphics[width=4cm]{images/logo.pdf}  
	\vspace{8mm}\\ 
	\LARGE\sffamily\textbf{\frauastitel}
	\vspace{8mm}\\ 
  \normalsize\sffamily{by}
	\vspace{4mm}\\ 
	\Large\sffamily{\frauasautor}
	\vspace{18mm}\\ 
	\LARGE\sffamily\textbf{\frauastyp}
	\vspace{2mm}\\ 
	\normalsize\sffamily{\frauasthesistype}
	\vspace{18mm}\\ 
	\normalsize\sffamily{\frauasstudiengangname}
	\vspace{2mm}\\ 
	\normalsize\sffamily{\frauasfakultaetlang}
	\vspace{2mm}\\ 
	\normalsize\sffamily{\frauaskoerperschaft}
	\vspace{18mm}\\ 
  \normalsize\sffamily{submitted at \frauasdatum}
	\vspace{8mm}\\   
	\normalsize\sffamily{\frauastutor : \ \frauasbetreuer}
	\vspace{2mm}\\ 
	\normalsize\sffamily{Second examiner: \frauaszweitkorrektor}
\end{center}
     

% Erklärung zur Eigenhändigkeit
\clearpage
%\setcounter{page}{1}
\thispagestyle{empty}
%\textsf{\large\textbf{Erklärung}}

\sffamily
\normalsize
\itshape

\chapter*{Declaration of Authorship}

We hearby certify that the project report We are submitting is entirely our own original work except where otherwise indicated.
We did not submit this work anywhere else before.
We are aware of the University's regulations concerning plagiarism, including those regulations concerning disciplinary actions that may result from plagiarism.
Any use of the works of any other author, in any form, is properly acknowledged at their point of use.

\begin{itemize}
  \item[$\circ$] We did not use any artificial intelligence-based writing tools nor any other artificial intelligence application, nor have we incorporated texts or sections generated by artificial intelligence applications into our written examination.
\end{itemize}

\noindent OR \textup{[Please tick the appropriate box]}

\begin{itemize}
  \item[$\circ$]{
    We used artificial intelligence-based writing tools and/or other artificial intelligence applications to write our assignment.
    We used these tools for guidance only.
    We confirm that most of the assignment is in our own words.
    We confirm that most of the assignment is in our own words. We have listed as tools all artificial intelligence applications used, including their product name and source (URL), and provided a complete overview of all functions used for the purpose of writing this examination.
    We have marked all texts and sections which were generated by and copied from artificial intelligence-based writing tools or other artificial intelligence applications accordingly.
  }
\end{itemize}

\upshape

\vspace{1cm}
\noindent \frauasort, \frauasdatum

\vspace{2cm}
\noindent Duc Xuan Bui - 158xxxx, Pham Minh Tuan Bui - 1589137,  Duc An Pham - 1584509

\vspace{2cm}
\noindent Hao Nhien Truong - 158xxxx, Hai Duong Tran - 1589111, Hoang Anh Khoa Nguyen - 10422037

% Inhaltsverzeichnis erzeugen
\cleardoublepage
\pdfbookmark{\contentsname}{Contents}
\tableofcontents
\cleardoublepage

% Den Hauptteil mit vergrößertem Zeilenabstand setzen
\onehalfspacing

% ------------------------------------------------------------------
% Add chapter tex files here


% ------------------------------------------------------------------

\label{lastpage}

% Neue Seite
\cleardoublepage 

% Literaturverzeichnis erzeugen
\begingroup
\cleardoublepage 
\let\clearpage\relax % Fix für leere Seiten (issue #25)
\printbibliography 
\endgroup

% Abkürzungsverzeichnis
\addchap{\frauasabbreviations}
\input{acronyms}

% Tabellenverzeichnis erzeugen
\cleardoublepage
\phantomsection
\addcontentsline{toc}{chapter}{\frauaslistoftables}
\listoftables

% Abbildungsverzeichnis erzeugen
\cleardoublepage
\phantomsection
\addcontentsline{toc}{chapter}{\frauaslistoffigures}
\listoffigures

% Listingverzeichnis erzeugen. Wenn Sie keine Listings haben,
% entfernen Sie einfach diesen Teil.
\cleardoublepage
\phantomsection
\addcontentsline{toc}{chapter}{\frauaslistings}
\lstlistoflistings

% Index ausgeben. Wenn Sie keinen Index haben, entfernen Sie einfach
% diesen Teil. Die meisten Abschlussarbeiten haben *keinen* Index.
\cleardoublepage
\phantomsection
\addcontentsline{toc}{chapter}{\frauasindex}
\printindex

% Anhang. Wenn Sie keinen Anhang haben, entfernen Sie einfach
% diesen Teil.
\appendix
\input{appendix} 

\end{document}
